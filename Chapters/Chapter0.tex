\chapter {Biểu đồ usecase}
\section{Tổng quan}
    \begin{tikzpicture}
    \tikzumlset{font=\scriptsize}

    \umlactor[x=0, y=2]{user 1}
    \umlactor[x=0, y=-2]{user 2}
    \umlactor[x=15, y=0]{database}

    % client system
    \begin{umlsystem}[x=2, y=0, fill=green!10]{client}
        \umlusecase[x=2, y=2, name=reg]{Đăng ký}
        \umlusecase[x=2, y=1, name=login]{Đăng nhập}
        \umlusecase[x=2, y=0, name=call]{Gọi}
        \umlusecase[x=2, y=-2.5, name=talk]{Nói chuyện}
        \umlusecase[x=2, y=-.75, name=receive]{Nhận cuộc gọi}
    \end{umlsystem}

    % server system
    \begin{umlsystem}[x=5.5, y=0, fill=red!10]{server}
        \umlusecase[x=5, y=2, name=createuser]{Tạo tài khoản mới}
        \umlusecase[x=5, y=1, name=authenticate]{Xác nhận người dùng}
        \umlusecase[x=5, y=0, name=connect]{Tạo kết nối đến người được gọi}
        \umlusecase[x=5, y=-1, name=receivetalkdata]{Nhận dữ liệu thoại}
        \umlusecase[x=5, y=-3, name=sendtalkdata]{Gửi dữ liệu thoại}
    \end{umlsystem}

    \umlinherit{user 2}{user 1} 
    \umluniassoc{user 1}{reg}
    \umlinclude{reg}{createuser}
    \umlassoc{createuser}{database}
    \umluniassoc{user 1}{login}
    \umlinclude{login}{authenticate}
    \umlassoc{authenticate}{database}
    \umluniassoc{user 1}{call}
    \umlinclude{call}{connect}
    \umlinclude{connect}{receive}
    \umluniassoc{receive}{user 2}
    \umlassoc{user 2}{talk}
    \umlassoc{talk}{user 1}
    \umlextend{talk}{receivetalkdata}
    \umlinclude{receivetalkdata}{sendtalkdata}
    \umlinclude{sendtalkdata}{talk}
\end{tikzpicture}


    \textbf{Giải thích}:
    Hệ thống gồm có 2 chương trình: \texttt{client} đặt ở máy của người sử dụng
    và \texttt{server} đặt ở máy chủ.

    Danh sách các \texttt{actor} và \texttt{usecase} như sau:
    \begin{table}[ht]
        \centering
        \begin{tabular}{| c | l |}
            \hline
            Actor & Ý nghĩa \\
            \hline
            user 1 & người sử dụng (gọi) \\
            user 2 & người sử dụng (nhận cuộc gọi) \\
            database & cơ sở dữ liệu, lưu thông tin về người dùng và các cuộc
            gọi \\
            \hline
        \end{tabular}
        \caption{Danh sách các \texttt{actor}}
    \end{table}

    \begin{table}[ht]
        \centering
        \begin{tabular}{| c | l |}
            \hline
            Usecase & Ý nghĩa \\
            \hline
            Đăng ký & Đăng ký tài khoản người dùng mới \\
            Đăng nhập & Đăng nhập bằng tài khoản đã có \\
            Gọi & Gọi cho một người dùng khác \\
            Nhận cuộc gọi & Nhận cuộc gọi từ một người dùng khác \\
            Nói chuyện & Nói chuyện với người dùng khác trong cuộc gọi \\
            \hline
            Tạo tài khoản mới & Tạo tại khoản người dùng mới \\
            Xác nhận người dùng & Xác nhận thông tin đăng nhập \\
            Tạo kết nối đến người được gọi & Tạo kết nối đến người được gọi \\
            Nhận dữ liệu thoại & Nhận dữ liệu tiếng nói từ người dùng \\
            Gửi dữ liệu thoại & Gửi dữ liệu tiếng nói đến người dùng khác \\
            \hline
        \end{tabular}
        \caption{Danh sách Usecase}
    \end{table}

    \textbf{Các kịch bản có thể có}
    \begin{itemize}
        \item Người dùng \texttt{user 1} đăng ký tài khoản mới bằng chương
            trình \texttt{client}: \texttt{client} sẽ yêu cầu \texttt{server}
            tạo môt tài khoản người dùng mới. Thông tin về tài khoản này sẽ
            được lưu vào \texttt{database}.
        \item Người dùng \texttt{user 1} đăng nhập bằng tài khoản của mình trên
            \texttt{ client}: \texttt{server} sẽ được yêu cầu xác nhận thông
            tin đăng nhập để \texttt{user 1} có thể dùng được các chức năng sau
            này. Để làm điều đó thì \texttt{server} phải truy vấn thông tin
            trong \texttt{database}.
        \item Người dùng \texttt{user 1} gọi cho một người dùng khác (chẳng hạn
            \texttt{user 2}. Khi đó \texttt{server} sẽ tạo kết nối đến chương
            trình \texttt{client} của người dùng \texttt{user 2}, yêu cầu
            \texttt{user 2} nhận cuộc gọi. Khi \texttt{user 2} đồng ý nhận cuộc
            gọi, 2 người dùng có thể nói chuyện được với nhau. Dữ liệu tiếng
            nói của 2 người có thể trao đổi trực tiếp, hoặc thông qua
            \texttt{server}.
    \end{itemize}
